%El autor de este texto es Oscar Abraham Olivetti Alvarez

\chapter*{Introducción}


\noindent Parte de la investigación científica se centra en buscar relaciones causales entre fenómenos y es una intuición común que usamos explicaciones causales para explicar fenómenos. Cuando hablamos de \textit{inferencias causales} nos referimos a la justificación de dichas relaciones y a la justificación en la que apelamos a relaciones causales. En términos de un argumento, son aquellos argumentos en los que en la conclusión aparece una relación causal entre dos fenómenos, o bien que las premisas involucran algún tipo de relación causal. El objeto de estudio de este trabajo son las inferencias causales en biología evolutiva. En particular, cómo la información causal puede ayudarnos a resolver algunos problemas que tiene la definición clásica de \emph{fitness}

La pregunta que queremos responder en este trabajo es: \textit{¿el modelo de Woodward es adecuado para tratar el fenómeno de la causalidad dentro de la biología evolutiva?} La pregunta es importante porque hay una tensión en torno a cómo interpretar la teoría de la selección natural. Este problema está reflejado en \cite{Bouchard2004} y \cite{Walsh2002}. Los primeros sostienen que la selección natural es a nivel de individuos y causal, mientras que los segundos defienden la tesis de que es a nivel de poblaciones y puramente estadística. Nosotros argumentaremos a favor de una interpretación causal.

Millstein en \citeyear{Millstein2006} trata de dar una salida al problema argumentando que la selección natural es un proceso a nivel de poblaciones, pero es un proceso causal. Sin embargo, Millstein no se avoca en su artículo a decirnos qué cuenta como proceso causal y qué no. Por lo que su solución es incompleta.

Diremos que esto se debe, tal como se menciona en el artículo de Uller y compañía \citeyear{Uller2020}, a que los biólogos se han preocupado poco por el problema de la causalidad y este estudio ha sido relegado a los filósofos de la ciencia. Debido a que cierto compromiso con causalidad indica qué evidencia sería apta para una hipótesis, es importante aclarar cómo se utiliza información para justificar hipótesis causales.

En este trabajo queremos defender la hipótesis de que el modelo de Woodward puede aclarar el uso de causalidad en biología. La causalidad es un concepto ampliamente usado para describir cómo se relacionan el medio ambiente y los organismos que habitan en él. La hipótesis es que el modelo de Woodward puede hacer más claro en términos metodológicos cómo se opera al experimentar cuando buscamos justificar hipótesis en biología. En particular hipótesis que tienen como consecuencia una relación causal. Este modelo tiene la virtud de que esclarecemos los compromisos ontológicos y metodológicos que se tienen al trabajar con hipótesis en selección natural.

El primer capítulo de este trabajo es el más expositivo y trata sobre cómo los modelos de explicación no han hecho justicia a cómo se trabaja en ciencias como la biología. Damos un breve repaso sobre las diferentes teorías de la explicación y proponemos al modelo de Woodward como una buena alternativa que rescata aspectos cruciales de la explicación en biología evolutiva, entre sus ventajas está el hecho de que este modelo de explicación no recurre a leyes. Además pretendemos esquematizar el modelo de Woodward, que consiste en que las explicaciones causales apelan a condicionales contrafácticos, y motivar que es un buen modelo porque ejemplifica y esclarece cómo se trabaja en biología evolutiva.

El segundo capítulo trata el tema de las leyes, particularmente la metafísica de las leyes. Si bien el modelo de Woodward puede capturar los aspectos epistémicos de la explicación a través de ofrecer una caracterización de causalidad, queda por discutir acerca del estatus de la causalidad y su relación con las leyes de la naturaleza. Nos interesa en particular argumentar a favor del indeterminismo causal. Los deterministas causales argumentan que dado un conjunto de variables, en conjunción con las leyes de la naturaleza, podemos predecir exactamente que pasará en el momento siguiente. Algún realista sobre las leyes puede argumentar que hay determinismo involucrado en el hecho de que haya leyes de la naturaleza\footnote{Aquí asumimos que el realista está comprometido con que las leyes son deterministas. Esta suposición no es extraña para el realista porque regularmente se caracteriza al realista como asumiendo el siguiente compromiso miodal: las leyes gobiernan los patrones del mundo. Si entendemos ``gobiernan'' como asumiendo un compromiso modal similar al de ``hacer necesario'', entonces el realista está comprometido con el determinismo.}. Las leyes describen estas conexiones necesarias. Esto es problemático para trabajar en biología porque, por lo regular, asumimos que hay indeterminismo. De manera que no somos capaces de predecir la evolución de las especies. Si causalidad y leyes están involucradas en el determinismo\footnote{Por supuesto no es necesario argumentar en favor del indeterminismo al nivel fundamental. El indeterminismo al nivel biológico puede deberse a nuestra ignorancia. De ser el caso, la indeterminación es a nivel epistémico. Sin embargo, sí es suficiente tener indeterminismo al nivel fundamental para argumentar a favor del indeterminismo biológico. La única manera de no aceptar esto esa aceptar la extraña tesis de que el nivel fundamental, digamos físico, es indeterminista, pero que a nivel biológico hay determinismo. Una tesis que parece poco plausible dados los argumentos expuestos en \cite{Graves1999}. Otra opción es ser agnósticos sobre este asunto como lo expone Millstein en \citeyear{pittphilsci4544}} entonces no es claro que podamos hablar de manera holgada sobre causalidad en biología.

El argumento en contra del determinismo causal es concentrarnos en la conjunción con las leyes de la naturaleza. Los necesitaristas argumentan que las leyes describen el comportamiento del mundo. En este sentido si las leyes describen el comportamiento del mundo y estas reflejan conexiones necesarias, entonces es suficiente con tener toda la información para afirmar que el mundo es determinista. Contra esto argumentamos a favor de un modelo antirrealista de las leyes. Argumentamos que las leyes son necesarias porque son derivaciones de axiomas. Pero esta conexión necesaria es una entre axiomas y teoremas, no una que esté en la naturaleza. Es decir, esta conexión necesaria no está dada por la metafísica sino por la lógica. Dicho todo esto podemos usar una aproximación como la presentada por Woodward para trabajar y modelar relaciones causales en biología evolutiva.

Por último, en el tercer capítulo queremos presentar cómo el modelo de Woodward, en conjunción con la práctica biológica, nos permite definir causalmente al \emph{fitness}. Ya en el primer capítulo presentamos al modelo de Woodward como una alternativa para hablar de explicación en biología. Lo anterior, en conjunción con el indeterminismo causal, nos permite hablar de causalidad en biología.

Hay un problema cuando definimos que el organismo más apto es aquél que deja más descendencia. Si el organismo más apto es el que deja más descendencia y aptitud está definido en términos de dejar más descendencia, entonces \emph{fitness} es tautológico. Rosenberg menciona que para librar a la selección natural de esta carga tautológica hay que definir al \emph{fitness} de otra manera: incorporando una noción de \emph{fitness} ecológico. Esta noción integra el ambiente en el que se desarrollan los organismos y cómo estos resuelven problemas de ``diseño''. Esto en principio nos libera de la carga tautológica ya que aptitud se define en términos de resolución de problemas. Son los organismos que mejor resuelven estos problemas los que dejan más descendencia. Esto nos permite leer causalmente el concepto de \emph{fitness} a la vez que podemos medirlo en términos de sus consecuencias, a saber, en el número de descendientes.

Sin embargo, la noción de diseño requiere un desarrollo ulterior, además de que no es claro cómo individuar los problemas de diseño a los que se refieren Bouchard y Rosenberg en \citeyear{Bouchard2004}. Para solventar esto, nosotros apelamos a los modelos experimentales utilizados para reunir evidencia. Supongamos, por ejemplo, que insertamos a un depredador en el medio ambiente. Pasa el tiempo y observamos que los organismos que son depredados han desarrollado extremidades más largas que les permiten moverse más rápido. Si en un diseño experimental intervenimos insertando el mismo depredador y obtenemos los mismos resultados, entonces podemos individuar exactamente cuál problema de ``diseño'' están resolviendo los organismos, al mismo tiempo que obtenemos una explicación del fenómeno según el modelo de Woodward: tienen extremidades más largas porque así pueden escapar con más facilidad. Si todo esto es correcto, obtenemos una noción más clara de \emph{fitness}.

La pregunta es importante porque en lo referente a la biología evolutiva, se considera que el medio ambiente en el que habita el organismo tiene influencia en la selección de caracteres, por ejemplo, el tamaño de órganos y extremidades. Esta influencia puede ser descrita en términos de una relación causal. Describir esta influencia de este modo, indica qué evidencia es pertinente para justificar el fenómeno a investigar. El modelo de Woodward nos parece adecuado por el énfasis que hace en la intervención, lo que casa bien con el hecho de que agentes concretos puedan realizar experimentos de laboratorio que sirvan para justificar las hipótesis que pueden generar al observar el entorno natural.



%El autor de este texto es Oscar Abraham Olivetti Alvarez
